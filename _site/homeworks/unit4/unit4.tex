%!TEX encoding = UTF-8 Unicode

\documentclass[11pt,largemargins]{homework}
\usepackage[utf8]{inputenc}
\usepackage{amsmath}
\usepackage{tikz}
\usepackage[french]{babel}


\newcommand{\hwname}{ENSA-Safi}
\newcommand{\hwemail}{}
\newcommand{\hwtype}{Travaux Dirigés}
\newcommand{\hwnum}{4}
\newcommand{\hwclass}{Prob} 
\newcommand{\hwlecture}{4}
\newcommand{\hwsection}{Z}



\begin{document}
\maketitle

\question*{Self-Indepenence}

Soit $A$ un évènement de l'espace d'étai $\Omega$.
\begin{arabicparts}
  \item Quand est ce $\mathbf{A}$ serait indépendant de soi $\mathbf{A}$
\end{arabicparts}

\question*{Modèle Conditionnel}
Dans une classe, la proportion des étudiants ayant préparé l'examen est $p \in
[0,1]$. Ceux qui n'ont pas prépayé l'examen réussissent avec une probabilité
égale a $\frac{1}{2}$ tandis que ceux qui l'ont préparé réussissent avec une
probabilité $\alpha \geq 0.99$.

\begin{arabicparts}
\item Si un étudiant échoue, quelle est la probabilité qu'il n'ait pas préparé
  l'examen? 
\end{arabicparts}

\question*{Indépendance deux a deux}
Votre voisine a deux enfants dont vous ignorez le sexe. On considère les trois
évènements suivants:

\begin{itemize}
  \item $A = $  {"\small les deux enfants sont de sexes différents"}.
  \item $B = $  {"\small L'ainée est une fille"}.
  \item $C = $  {"\small Le cadet est un garçon"}.
\end{itemize}
    On suppose que la probabilité d'avoir une fille est $p = \frac{1}{2}$.\\
    \begin{arabicparts}
        \item Montrer que les trois événement $A$ , $B$ et $C$ sont deux a
          deux indépendants.
        \item Est ce qu'ils sont mutuellement indépendants?
    \end{arabicparts}


\question*{Fiabilite}
On suppose que les unités d'un système peuvent être fonctionnelles avec un
probabilité $\frac{2}{3}$ et peuvent échouer avec une probabilité $\frac{1}{3}$.
On suppose que ces unités sont \textbf{indépendants}. Pour chaque système
calculer la probabilité que le système soit fonctionnel( i.e. Il existe un
chemin du point a gauche a celui de droite).


\begin{arabicparts}
    \item Premier système: 
      \begin{center}
      \begin{tikzpicture}[scale=1, transform shape]
       \node[draw, inner sep=0pt, minimum size=12pt, thick]  (A) at (0,0) {}; 
       \node[draw, inner sep=0pt, minimum size=12pt, thick]  (B) at (2,1) {}; 
       \node[draw, inner sep=0pt, minimum size=12pt, thick]  (C) at (2,-1) {}; 
       \path[draw,thick] (-1,0) -- (A) -- (1,0) --
         (1,1)--(B)--(3,1)--(3,0)--(4,0);

       \path[draw,thick] (-1,0) -- (A) -- (1,0) --
         (1,-1)--(C)--(3,-1)--(3,0)--(4,0);
      \end{tikzpicture}
      \end{center}

      \item Deuxième systeme:
        \begin{center}
        \begin{tikzpicture}[scale=1, transform shape]
       \node[draw, inner sep=0pt, minimum size=12pt, thick]  (A) at (1,1) {}; 
       \node[draw, inner sep=0pt, minimum size=12pt, thick]  (B) at (3,1) {}; 
       \node[draw, inner sep=0pt, minimum size=12pt, thick]  (C) at (2,-1) {}; 
       \path[draw,thick] (-1,0)--(0,0)--(0,1)--(A)--(B)--(4,1)--(4,0)--(5,0);
       \path[draw,thick] (-1,0)--(0,0)--(0,-1)--(C)--(4,-1)--(4,0)--(5,0);
        \end{tikzpicture}
        \end{center}
        

\end{arabicparts}




\end{document}
