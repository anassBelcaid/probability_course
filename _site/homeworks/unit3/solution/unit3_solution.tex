%%% Laboratory	 Notes
%%% Template by Mikhail Klassen, April 2013
%%% Contributions from Sarah Mount, May 2014
\documentclass[a4paper]{tufte-handout}
\usepackage{tikz}

\newcommand{\workingDate}{\textsc{April $|$ 2022}}
\newcommand{\userName}{A.Belcaid}
\newcommand{\institution}{ENSA-Safi}

\usepackage{lab_notes}

\usepackage{hyperref}
\hypersetup{
    pdffitwindow=false,            % window fit to page
    pdfstartview={Fit},            % fits width of page to window
    pdftitle={Correction TD3},     % document title
    pdfauthor={A.Belcaid},         % author name
    pdfsubject={},                 % document topic(s)
    pdfnewwindow=true,             % links in new window
    colorlinks=true,               % coloured links, not boxed
    linkcolor=DarkScarletRed,      % colour of internal links
    citecolor=DarkChameleon,       % colour of links to bibliography
    filecolor=DarkPlum,            % colour of file links
    urlcolor=DarkSkyBlue           % colour of external links
}


\title{Solution TD 3}
\date{2022}

\begin{document}
\maketitle

\renewcommand{\P}{\mathbf{P}}

\section{Exercice 1}

\begin{enumerate}
  \item \textbf{Correct}, car les elements dans $B$ maintienent la meme
    proportion que dans la loi originale.\\
  \item \textbf{Fausse}, car les elements qui ne sont pas dans $B$ possedent une
    probabilite nulle!
\end{enumerate}

\section{Exercice 2}
\begin{enumerate}
  \item Calculer la probabilite de trouver un \textbf{double}.
    $$
    \P(D=\text{\small obtenir un doubble}) = \frac{6}{36} = \frac{1}{6}
    $$
  \item  On note l'evenement $S=$"somme $leq$ 4".
    $$
    \P(D | S) = \frac{\P(D\cap S)}{\P(S)} = \frac{\frac{2}{36}}{\frac{6}{36}}=
    \frac{1}{3}.
    $$
  \item Soit l'evenement $S=$ "{\small obtenir 6 dans l'un des deux des}"
    Alors il suffit de compter toutes les paires pour obtenir que:

    $$
    \P(S) = \frac{11}{36}
    $$
  \item Calculer la probabilite d'obtenir un $6$ sachant qu'un on as pas un
    double.

    $$
    \P(S | D^c) = \frac{\P(S \cap D^c)}{\P(D^c)} =
    \frac{\frac{10}{36}}{\frac{30}{36}} = \frac{1}{3}
    $$
\end{enumerate}
\section{Exercice 3}
On note $A_i$ l'evenement de selection de la piece de monnaie $i$. En utilisant
lsda loi de probabilite totale for l'infinite des scenarios, on obtient:

\begin{eqnarray*}
  \P(\text{\small Pile})  & = & \sum_{i=1}^\infty \P(A_i)\P(\text{\small
  Pile}\;|\;A_i)\\
                          & = & \sum_{i=1}^\infty 2^{-i}3^{-i}\\
                          & = & \frac{\frac{1}{6}}{1 - (\frac{1}{6})}\\
                          & = & \frac{1}{5}
\end{eqnarray*}

\section{Exercice 4}
On note $A$ l'evenement qu'une personne soit malade. et $B$ que le test soit
positif.\\

\begin{enumerate}
  \item 
    \begin{eqnarray*}
      \P(B) & = & \P(A)\P(B\;|\;A)  + \P(A^c)\P(B\;|\;A^c)\\
            & = & 0.001\cdot 0.95\;+\; 0.999\cdot 0.05\\
            & = & 0.0509
    \end{eqnarray*}
  \item 
 \begin{eqnarray*}
   \P(A\;|\;B) &=& \frac{\P(A)\P(B\;|\;A)}{\P(B)} \\
               &=& \frac{0.001\cdot 0.95}{.0509}\\
               &=& 0.01866
 \end{eqnarray*}

  { \scriptsize
   On remarque que meme si la precision de ce test est tres elevee. Une personne
   teste positive a peu de chance d'etre malade ($2\%$). L'explication est la
   frequence de la maladie qui est de $\frac{1}{1000}$. On s'attend que
   $1000 \cdot 0.999\cdot 0.05\approx 50$ de tester positive sans qu'ils soit
   malades!!
 }
 
\end{enumerate}

\hrule

\section{Exercice 5}
\begin{enumerate}
  \item
\begin{enumerate}
  \item La reponse est \textbf{non}.\\
    On note 
    \begin{itemize}
      \item $A$: l'evenement que la somme est $10$.
      \item $C$: l'un des deux des est $5$.
    \end{itemize}
    On as:
    $$
    \P(A) = \frac{1}{25} \quad \text{ et }\quad \P(C) = \frac{5}{10}
    $$
    Or on sait que 
    $$
    \P(A\cap C) = \P(A) = \frac{1}{25} \neq \P(A)\cdot\P(C)
    $$
  \item Aussi c'est \textbf{Faux}.\\
    On note:
    \begin{itemize}
      \item $D$: L'evenement que l'un des resultat est $1$.
    \end{itemize}
    on as 
    $$
    \P(D) = \frac{9}{25}
    $$

    $$
    \P(A\cap D) = \P(\emptyset) = 0  \neq \P(A)\cdot\P(D)
    $$
\end{enumerate}
\item Suivre la meme analyse, on note $B=\{\text{\small la somme est 8}\}$ et
  $E=\{\text{\small obtenir un double}\}$
  $$
  \P(B) = \frac{3}{25} \quad \text{ et } \quad \P(E) = \frac{1}{5}
  $$

  On as alors
  $$
  \P(B\cap E) = \P(\{(4,4)\}) = \frac{1}{25}\neq \P(B)\cdot \P(E)
  $$
\end{enumerate}
\end{document}
