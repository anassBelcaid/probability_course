\documentclass[palatino,code]{ensaexam}
%http://www.jaicompris.com/lycee/math/probabilite/loi-binomiale/loi-binomiale.php
%http://www.jaicompris.com/lycee/math/probabilite/variable-aleatoire/variable-aleatoire.php

\begin{document}
\ModuleName{Probabilite}
\ExamCode{CP224}
\ExamPeriod{Spring 2022}
\TimeAllowed{120}
\Logo{
\begin{center}
  \includegraphics[width=3cm, height=3cm]{ENSA-SAFI.png}
\end{center}
}
\Instructions{
  \begin{itemize}
    \item Vous avez {\bf\TheTimeAllowed\; minutes}. 
    \item Vérifier que vous disposez de toutes les pages. 
    \item L'échange d'outils est strictement \textbf{interdit}.
  \end{itemize}
 
}
\MakeHeading
\vspace*{1cm}

%% For questions use the command 
%% \begin{questions} 
%% \questoin[grade]
%% \end{questions}



 \begin{questions}
% Random variable {{{ 
   \titledquestion{(**) Question 1}
   On considère une variable aléatoire $\mathbf{X}$ qui a une loi de
   probabilité donnée par:
   \begin{equation*}
     \mathbf{P}_X(x) =
    \begin{cases}
      kx & x = 2, 4, 6\\[4pt]
      k(x-2) & x =8\\[4pt]
      0      & \text{sinon}
   \end{cases}
   \end{equation*}
   où $k$ est une constante.
\begin{parts}
  \part[1] Déterminer la valeur de $k$.
  \part[1] Calculer $\mathbf{P}_X(x < 5)$.
  \part[1] Calculer l'espérance de $X$.
  \part[1] Donner la valeur de $E[X^2]$.
  \part[1] Calculer la valeur de $\text{Var}(3 - 4X)$.\\[8pt]
\end{parts}
% }}} % Random variable %

% Fonction de répartition {{{ %
\titledquestion{(**) Question 2}
On considère une variables aléatoire $X$ qui peut prendre seulement les trois
valeurs 1, 2, et 3. Pour chacune des ces valeurs, on définit la fonction de
\textbf{répartition}:

\begin{equation*}
  F(t) = \mathbf{P}_X(x < t) = \frac{t^3 + k}{40}\quad t=1,2,3
\end{equation*}

\begin{parts}
  \part[1]  Déterminer la valeur de $k$. 
  \part[2] Donner la loi de probabilité de $X$.
  \part[2] Sachant que $\text{Var}(X) = \frac{259}{320}$, calculer la valeur de
  $\text{Var}(4X-5)$.\\[12pt]
\end{parts}

% }}} Fonction de repartition %

% Covariance et Bernoulli {{{ %
  \titledquestion{(***) Question 3}
  Une urne contient $6$ boules blanches et $n$ boules rouges ($n$ est un nombre
  entier tel que $n\geq 0$) toutes indiscernables au toucher. Un joueur tire au
  hasard, successivement et sans remise, deux boules de l'urne. Pour chaque
  boule blanche tirée, il gagne $2$ et pour chaque boule rouge, il perd $3$.

  \begin{parts}
    \part[1] Quelles sont les différentes valeurs que peut prendre $X$. 
    \part[1] Montrer que 
       \begin{equation*}
         \mathbf{P}_X(X = -1) = \frac{12n}{(n+6)(n+5)}
       \end{equation*}
    \part[2] Déterminer la loi de probabilité de $X$.
    \part[1] Montrer que 
    \begin{equation*}
      \mathbf{E}(X) = \frac{-6\left(n^2 + n -20\right)}{(n+6)(n+5)}
    \end{equation*}
  \end{parts}
  
% }}} Covariance et Bernoulli %

% Velo {{{ %
\titledquestion{(***) Question 4}
Un élève se rend a vélo au lycée distant de $3$ km de son  domicile à une
vitesse constante de $15$ km/h. Sur  le parcours, il rencontre $6$ feux
tricolores non synchronisés. Pour chaque feu,  la probabilité qu'il soit au vert
est $\frac{2}{3}$. Un feu orange ou rouge lui fait perdre une minute et demie.\\

On appelle $\mathbf{X}$ la variable aléatoire correspondant au nombre  de feux verts
rencontrés par l'élève sur son parcours et $\mathbf{T}$  la variable aléatoire
donnant le temps  en minutes mis par l'élève pour se rendre au lycée.

\begin{parts}
  \part[2] Déterminer la loi de probabilité de $\mathbf{X}$.
  \part[1] Exprimer $\mathbf{T}$ en fonction $\mathbf{X}$.
  \part[1] En déduire $\mathbf{E}(T)$ et interpréter.
  \part[1] L'élève part $17$ minutes avant  le début des cours. Déterminer  la
  probabilité  qu'il arrive en retard.
\end{parts}
% }}} Velo %

 \end{questions}

\end{document}

