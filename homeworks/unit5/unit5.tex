%!TEX encoding = UTF-8 Unicode

\documentclass[11pt,largemargins]{homework}
\usepackage[utf8]{inputenc}
\usepackage{amsmath}
\usepackage{multicol}


\newcommand{\hwname}{ENSA-Safi}
\newcommand{\hwemail}{}
\newcommand{\hwtype}{Travaux Dirigés : Probabilite}
\newcommand{\hwnum}{1}
\newcommand{\hwclass}{5}
\newcommand{\hwlecture}{ 5}
\newcommand{\hwsection}{Z}



\begin{document}
\maketitle

\question*{Mots d'un alphabet}

On vous donne l'ensemble des lettres suivantes $\{A,B,C,D,E\}$.
\begin{arabicparts}
  \item Combien de mots de trois lettres peut on construire a partir de cet
    alphabet en utilisant chaque lettre qu'une seule fois?
  \item Combien de sous ensembles peut avoir de cet alphabet?
  \item Combien de mots (de cinq lettres) on peut construire si on ne peut
    utiliser une lettre qu'une seule fois et que les lettres $A$ et $B$ doivent
    être voisins. Un exemple de ces mots est donnes par:

    
    \begin{multicols}{3}
     \begin{itemize}
     \item "ABCDE" 
     \item "EBADC"
     \item "CEDAB"
     \end{itemize}
   \end{multicols}
 \item Quelle est la probabilité de construire un mot décrit dans  la question
   \textbf{1.3}.
\end{arabicparts}

\question*{Anecdode d'anniversaire}


On suppose que $n$ personnes assistent a une fête. On suppose que la probabilité
qu'une personne soit ne dans une date est uniforme. On ignore aussi la
complication des années bissextiles.\footnote{Personne n'est ne 29 fevrier}.\\

\begin{arabicparts}
    \item Calculer la probabilité que chaque personne possède un date
      d'anniversaire différente des autres?

\end{arabicparts}


\question*{Tours dans un echequier}
On considère le cas de $\mathbbf{8}$ tours dans un $8 \times 8$ échiquier. Ou
tous les emplacements sont équiprobables.\\
\begin{arabicparts}
    \item Calculer la probabilité que tous les tours ne s'attaquent pas. On
      rappelle qu'une tour peut attacher sa ligne et colonne.
\end{arabicparts}
\end{document}
