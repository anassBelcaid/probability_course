%!TEX encoding = UTF-8 Unicode

\documentclass[11pt,largemargins]{homework}
\usepackage[utf8]{inputenc}
\usepackage{amsmath}


\newcommand{\hwname}{ENSA-Safi}
\newcommand{\hwemail}{}
\newcommand{\hwtype}{Travaux Dirigés : Variable Aleatoire}
\newcommand{\hwnum}{6}
\newcommand{\hwclass}{ Prob}
\newcommand{\hwlecture}{ 6 }
\newcommand{\hwsection}{Z}



\begin{document}
\maketitle

\question*{Loi de probabilite}

Soit la variable aleaotoire $\mathbf{X}$  et sa loi de probabilite:

\begin{equation}
  P_X(x) = \begin{cases}
    \frac{x^2}{a},& \text{ pour } x \ in \{-3,-2,-1,1,2,3\}\\[4pt]
    0             & \text{sinon}
  \end{cases}
\end{equation}
ou $a > 0$ est un parametre reel.

\begin{arabicparts}
    \item Determiner la valeur de $a$.
    \item Donner la loi de probabilite de la varaible aleatoire $Z = X^2$.
\end{arabicparts}

\question*{Loi d'un de truque}
On considère un dé cubique truqué dont les faces sont numérotés de $1$ à $6$ et
on note $X$ la variable aléatoire donnée par le numéro de la face du dessus. On suppose que le dé est truqué de sorte que la probabilité d'obtenir une face est proportionnelle au numéro inscrit sur cette face.


\begin{arabicparts}
\item Determiner la loi de $X$.
\item Calculer son esperance.
\item On pose $Y = \frac{1}{X}$, determiner la loi de $Y$ et son esperance.
  
\end{arabicparts}


\end{document}
