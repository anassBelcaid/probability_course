%!TEX encoding = UTF-8 Unicode

\documentclass[11pt,largemargins]{homework}
\usepackage[utf8]{inputenc}
\usepackage{amsmath}


\newcommand{\hwname}{ENSA-Safi}
\newcommand{\hwemail}{}
\newcommand{\hwtype}{Travaux Dirigés : Variable Aléatoire}
\newcommand{\hwnum}{6}
\newcommand{\hwclass}{ Prob}
\newcommand{\hwlecture}{ 6 }
\newcommand{\hwsection}{Z}



\begin{document}
\maketitle

\question*{Loi de probabilité}

Soit la variable aléatoire $\mathbf{X}$  et sa loi de probabilité:

\begin{equation}
  P_X(x) = \begin{cases}
    \frac{x^2}{a},& \text{ pour } x \ in \{-3,-2,-1,1,2,3\}\\[4pt]
    0             & \text{sinon}
  \end{cases}
\end{equation}
ou $a > 0$ est un paramètre réel.

\begin{arabicparts}
    \item Déterminer la valeur de $a$.
    \item Donner la loi de probabilité de la variable aléatoire $Z = X^2$.
\end{arabicparts}

\question*{Loi d'un dé truqué}
On considère un dé cubique truqué dont les faces sont numérotés de $1$ à $6$ et
on note $X$ la variable aléatoire donnée par le numéro de la face du dessus. On suppose que le dé est truqué de sorte que la probabilité d'obtenir une face est proportionnelle au numéro inscrit sur cette face.


\begin{arabicparts}
\item Déterminer la loi de $X$.
\item Calculer son espérance.
\item On pose $Y = \frac{1}{X}$, déterminer la loi de $Y$ et son espérance.
  
\end{arabicparts}


\question*{Cles}

Une personne a quatre clés mais \textbf{une seule} ouvre la porte. Elle les
essaie au hasard en éliminant celles qui ne fonctionnent pas.\\

Soit \textbf{X} "Le nombre d'essais pour ouvrir la porte" qui est une variable
aléatoire.

\begin{arabicparts}
    \item Déterminer la loi de probabilité de $X$.
    \item Calculer $\mathbf{E}(X)$ et $\mathbf{Var}(X)$.
\end{arabicparts}


\question*{Loi de Pacal}

On considère une série d'épreuves indépendants. A chaque  épreuve, on observe un
\textbf{succes} avec une probabilité $\mathbf{p}$ et un échec avec une probabilite
$\mathbf{1-p}$.\\

On considère la variable aléatoire $\mathbf{X}$:

$X$ = "nombre d'épreuves nécessaires pour obtenir le premier succès".\\

\begin{arabicparts}
    \item Donner la loi de probabilité de $X$.
    \item Vérifier que $\mathbf{E}(X) = \frac{1}{p}$.
    \item Vérifier la propriété d'\textbf{absence de mémoire}
      $$
      \mathbf{P}(X > k\;|\; X > j) = \mathbf{P}(X >  k-j )\quad k > j
      $$
    \item Vous décidez de vendre votre maison et d'accepter  la premiere offre
      d'achat supérieure a $\mathbf{K}$ DH. ou $K$ est fixe. On suppose que les
      offres d'achats sont des variables aléatoires indépendants dont on suppose
      la loi de probabilité est connue.\\

      Soit $\mathbf{N}$ la variables aléatoire qui représente le nombre
      d'offres reçues avant de vendre la maison.

      \begin{arabicparts}
      \item Donner la loi de probabilite de $\mathbf{N}$.
      \item Donner son esperance?
      \end{arabicparts}
\end{arabicparts}


\end{document}
