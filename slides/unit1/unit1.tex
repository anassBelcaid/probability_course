%{{{
\documentclass{beamer}
\usetheme{ensam}
\usepackage{pgfplots}
\usepackage{subcaption}
\usepackage{acronym}
\usepackage{tikz}
\usetikzlibrary{calc, patterns, shapes.geometric}
\usepackage{amsmath}
\usepackage {algorithmic}
\usepackage{algorithm} \usepackage{eqparbox}
\usepackage[font=scriptsize]{caption}
\usetikzlibrary{bayesnet,positioning,calc}
\tikzstyle{obs} = [latent,fill=lightBlue]
\tikzstyle{default}=[draw=sexyRed,thick,rounded corners,text width=0.5in,font=\scriptsize,align=center]
\usepgfplotslibrary{colorbrewer}
\definecolor{ForestGreen}{RGB}{34,139,34}
\newcommand{\comment}[1]{\textcolor{ForestGreen}{#1}}
%algorithmic comment
\renewcommand\algorithmiccomment[1]{%
  \hfill\comment{\#\scriptsize\eqparbox{COMMENT}{#1}}%
}
\title{Probabilité : Science d'Incertitude et de Données.}
\author{\underline{A.Belcaid}}
\institute{\small ENSA-Safi} 

%tikz bayesian theme
\usetikzlibrary{bayesnet,positioning,calc}
\tikzstyle{obs} = [latent,fill=lightBlue]
\tikzstyle{default}=[draw=sexyRed,thick,rounded corners,text width=0.5in,font=\scriptsize,align=center]
\DeclareMathOperator{\argmin}{argmin}

\pgfplotsset{every tick label/.append style={font=\tiny}}



% add bibliography
\usepackage[style=authoryear]{biblatex}
\renewcommand*{\nameyeardelim}{\addcomma\addspace}
%}}}
% Title page {{{ %

\begin{document}
\maketitle

%{{{ Table of contents
\begin{frame}
\tableofcontents
\end{frame}
%}}}

% Introduction {{{ %
\section{Introduction}

% Definition Probabiilite {{{ %
\subsection{Definition}
\begin{frame}[<+->]
    \frametitle{Théorie des probabilités}

    \begin{block}{Définition}
        Une théorie mathématique pour quantifier le \alert{HASARD}.
    \end{block}

Elle est \textbf{fondamentale} dans de nombreux domaines d'applications:

\begin{itemize}
    \small
    \item \textbf{Physique}: (physique quantique, physique des
        particules)\\[.2cm]
    \item \textbf{Informatique} et Réseaux de Télécommunications.\\[.2cm]
    \item \textbf{Traitement de signal et de la parole}.\\[.2cm]
    \item \textbf{Biologie, Écologie}\\[.2cm]
    \item \textbf{Économie, Finance, Assurance}\\[.2cm]
        
\end{itemize}

\end{frame}
% }}} Definition Probabiilite %
% Modelisation Abstraction Simulationj {{{ %
\subsection{Modelisation Abstraction}
\begin{frame}[<+->]
    \frametitle{Modélisation, Abstraction, Simulation}
    
    \begin{itemize}
        \item \structure{Innombrable situations ou le hasard intervient}

            \begin{block}{Abstraction}
                \small
                Nécessité d'une \alert{\textbf{abstraction mathématique}}  pour donner un cadre général d'étude.
            \end{block}
            \begin{center}
                \textbf{Modèle probabiliste}.
            \end{center}
        \item Aussi utilise pour des fins \textbf{fin numeriques}:
            \alert{\textbf{Méthodes de
        Monte-Carlo.}}
        \vspace{1cm}
         \begin{itemize}
             \item Efficace en \textbf{\structure{grande dimension}}.\\[.5cm]
             \item Simulation de \alert{{phénomènes irréguliers.}}
         \end{itemize}
    \end{itemize}
\end{frame}
% }}} Modelisation Abstraction Simulationj %
% Objectif cours {{{ %
\subsection{Objectif cours}
\begin{frame}[<+->]
    \frametitle{Objectif du cours}
    \small
    Le but  de ce cours est de vous donner une approche \textbf{élémentaire} mais
    \alert{rigoureuse} de la théorie probabiliste et de l'illustrer avec un
    certain nombre de simulations.
    \vspace{.5cm}
    
    A la fin du cours, on souhaite vous transmettre une démarche de mathématique
    appliqués, qui se décrit par les trois étapes suivantes:

    \vspace{.5cm}
    \begin{itemize}
        \item Modélisation.
          \item Résolution théorique
          \item Expérimentation numérique
    \end{itemize}
\end{frame}
% }}} Objectif cours %
% }}} Introduction %

% Experience/evenement Aleatoire {{{ %
\section{Expérience Évènement Aléatoire}
% Espace d'etats {{{ %
\subsection{Espace d'états}
\begin{frame}[<+->]
    \frametitle{Espace d'états}
    
    \begin{block}{Expérience aléatoire}
        \scriptsize
        On appelle une \textbf{\alert{Expérience aléatoire}} une expérience
        $\xi$, qui sous \textbf{conditions identiques}, peut reproduire
        \alert{\textbf{plusieurs résultats possibles}} qui sont
        \textbf{\alert{imprévisible}}\footnote{On ne peut pas prédire leur
            résultat par avance}.
    \end{block}
    \vspace{.5cm}
    \begin{itemize}
        \item Pour  décrire une telle expérience, il nous faut:
            \begin{itemize}
                \small
                \item \alert{Décrire tous les résultats possibles}\\[.2cm]
                \item  Décrire notre croyance pour chaque résultat.
            \end{itemize}
    \end{itemize}
\end{frame}
% }}} Espace d'etats %
% Espace detat definitionj {{{1 %
\begin{frame}[<+->]
    \frametitle{Espace d'états}
    \begin{columns}
    \column{0.5\textwidth}        

    \begin{itemize}
        \small
        \item 
        \begin{block}{Espace d'État}
            \scriptsize
           Note $\alert{\mathbf{\Omega}}$ espace de tous les résultats possibles.
        \end{block}

    \item  Les évènements doivent être:
        \begin{itemize}
            \item Mutuellement \alert{exclusive}.
            \item Collectivement \structure{exhaustive}
            \item Au niveau correct de \structure{détail}?.
        \end{itemize}
    \end{itemize}

    \column{0.5\textwidth}        
\begin{figure}[htpb]
\begin{center}
\begin{tikzpicture}[scale=1, transform shape]
    \draw[sexyRed,thick] (0,0) rectangle (4,2);
    \node at (0.2,0.2) {$\Omega$};
    \node[fill,circle,  minimum size=3pt, inner sep=0pt] at (0.2, 1){};
    \node[fill,circle,  minimum size=3pt, inner sep=0pt] at (0.7, 1.5){};
    \node[fill,circle,  minimum size=3pt, inner sep=0pt] at (1.2, 1.8){};
    \node[fill,circle,  minimum size=3pt, inner sep=0pt] at (3, .2){};
    \node[fill,circle,  minimum size=3pt, inner sep=0pt] at (3.5, .9){};
    \node[fill,circle,  minimum size=3pt, inner sep=0pt] at (2, .9){};
    \node[fill,circle,  minimum size=3pt, inner sep=0pt] at (1.9, .5){};
    \node[fill,circle,  minimum size=3pt, inner sep=0pt] at (.5, .4){};
\end{tikzpicture}
\end{center}
\end{figure}

\begin{figure}[htpb]
\begin{center}
\begin{tikzpicture}[scale=0.8, transform shape]
    \draw[sexyRed,thick] (0,0) rectangle (2,1);
    \node at (2.2,1.2) {$\Omega$};
    \node[fill,circle,  minimum size=3pt, inner sep=0pt, label=right:T] at (0.2, .2){};
    \node[fill,circle,  minimum size=3pt, inner sep=0pt, label=right:H] at (0.2, .8){};

\end{tikzpicture}
\end{center}
\end{figure}


\begin{figure}[htpb]
\begin{center}
\begin{tikzpicture}[scale=0.8, transform shape]
    \draw[sexyRed,thick] (0,0) rectangle (4,2);
    \node at (4.2,2.2) {$\Omega$};
    \node[fill,circle,  minimum size=3pt, inner sep=0pt, label=right:{T et Pluie}] at (0.2, .2){};
    \node[fill,circle,  minimum size=3pt, inner sep=0pt, label=right:{T et Non
            pluie}] at (0.2, .6){};
    \node[fill,circle,  minimum size=3pt, inner sep=0pt, label=right:{H et Pluie}] at (0.2, 1){};
    \node[fill,circle,  minimum size=3pt, inner sep=0pt, label=right:{H et non
        Pluie}] at (0.2,1.4){};

\end{tikzpicture}
\end{center}
\end{figure}

    \end{columns}
    
\end{frame}
% }}} Experience/evenement Aleatoire %
% Exercice Espace d'etats {{{ %
\begin{frame}[<+->]
    \frametitle{Mini Exercice}
    
    \begin{itemize}
        \item Pour l'expérience de lancer un de, pour chaque choix,
            \textbf{Déterminer} si on possède un espace d'états correct.
            \begin{itemize}
                \item
                    \begin{equation}
                        \Omega = \{\text{H et pluie}, \text{H et non pluie},
                        \text{T} \}
                    \end{equation}

                \item
                    \begin{equation}
                        \Omega = \{\text{H et pluie}, \text{T et non pluie},
                        \text{T} \}
                    \end{equation}
            \end{itemize}
    \end{itemize}

\end{frame}
% }}} Exercice Espace d'etats %
% Espace d'etats fini /discret {{{ %
\subsection{discret fini}
\begin{frame}[t]
    \frametitle{Espace d'états : Discret/fini}
    
    \begin{block}{Exemple}
        Lance de deux de a \textbf{quatre} faces.
    \end{block}

    \begin{figure}[htpb]
    \begin{center}
    \begin{tikzpicture}[scale=.7, transform shape]
        \draw[thick, black] (0,0) -- (4,0);
        \draw[thick, black] (0,1) -- (4,1);
        \draw[thick, black] (0,2) -- (4,2);
        \draw[thick, black] (0,3) -- (4,3);
        \draw[thick, black] (0,4) -- (4,4);
        \node at (0.5,-.5) {1};
        \node at (1.5,-.5) {2};
        \node at (2.5,-.5) {3};
        \node at (3.5,-.5) {4};


        \draw[thick, black] (0,0) -- (0,4);
        \draw[thick, black] (1,0) -- (1,4);
        \draw[thick, black] (2,0) -- (2,4);
        \draw[thick, black] (3,0) -- (3,4);
        \draw[thick, black] (4,0) -- (4,4);

        \node at (-.5,0.5) {1};
        \node at (-.5,1.5) {2};
        \node at (-.5,2.5) {3};
        \node at (-.5,3.5) {4};
        
        % quelque resultat
        \node[sexyRed] at (0.5,0.5) {1,1};
        \node[sexyRed] at (2.5,1.5) {3,2};
        \node[sexyRed] at (1.5,2.5) {2,3};

            
        % second represetnation

         \node[fill,circle,  minimum size=3pt, inner sep=0pt] (R)at (7, 0){};
         \node[fill,circle,  minimum size=3pt, inner sep=0pt,label=above:$2$]
             (X1) at (8, .5){};
         \node[fill,circle,  minimum size=3pt, inner sep=0pt,label=above:$1$]
             (X2) at (8, 1.5){};
         \node[fill,circle,  minimum size=3pt, inner sep=0pt,label=above:$3$]
             (X3) at (8, -.5){};
         \node[fill,circle,  minimum size=3pt, inner sep=0pt, label=above:$4$]
             (X4) at (8, -1.5){};

         \edge[-]{R}{X1,X2,X3,X4}

         % second part

         \node[fill,circle,  minimum size=3pt, inner sep=0pt,label=right:{$1,1$}]
             (X11) at (9, 2.3){};
         \node[fill,circle,  minimum size=3pt, inner sep=0pt,label=right:{$1,2$}]
             (X12) at (9, 2){};
         \node[fill,circle,  minimum size=3pt, inner sep=0pt,label=right:{$1,3$}]
             (X13) at (9, 1.7){};
         \node[fill,circle,  minimum size=3pt, inner sep=0pt, label=right:{$1,4$}]
             (X14) at (9, 1.4){};
        
         \edge[-]{X2}{X11,X12,X13,X14}
         \node[lightBlue] at (8,-2){Modèle arbre};
    \end{tikzpicture}
    \end{center}
    \end{figure}
    
\end{frame}
% }}} Espace d'etats fini /discret %
% continuous space {{{ %
\subsection{Example continue}
\begin{frame}[t]
    \frametitle{Exemple continue}

    \begin{itemize}
        \item Lancé d'une balle dans une \textbf{position} $(x,y)$ tel que
            $0\leq x,y\leq 1$.
    \end{itemize}
    
    \begin{figure}[htpb]
    \begin{center}
    \begin{tikzpicture}[scale=1, transform shape]
        
        \path[draw,thick,-stealth] (0,0)--(5,0)node[pos=1,label=below:$x$]{};
        \path[draw,thick,-stealth] (0,0)--(0,5)node[pos=1,label=below:$y$]{};
        \node (A) at (4,-0.2) {$1$};
        \node (B)at (-0.2,4) {$1$};
        \path[draw, thick] (A) -- (4,4) -- (B);
        
         \node[fill,circle,  minimum size=3pt, inner sep=0pt, sexyRed ]
             (X14) at (2.4, 1.4){};
        
    \end{tikzpicture}
    \end{center}
    \caption{Exemple d'espace d'étai continus}%
    \label{fig:}
    \end{figure}
    
\end{frame}

% }}} continuous space %

% Axiom de probabilite {{{ %
\subsection{Axiome de probabilité}
\begin{frame}[t]
   \frametitle{ Evenement aleatoire}
    
   \begin{block}{Évènement}
       
       On appelle un \textbf{\alert{Évènement aléatoire}} $\mathbf{A}$ un sous
       ensemble de l'ensemble d'étai $\xi$ 
   \end{block}
   \begin{itemize}
       \item On associe des \textbf{probabilités} a des évènements.
           \end{itemize}
        \pause
\begin{figure}[htpb]
\begin{center}
\begin{tikzpicture}[scale=0.8, transform shape]
    \path[draw, thick] (0,0) -- (4,0) -- (4,4) -- (0,4) --cycle; 
    \node at (-.2,0) {$0$};
    \node at (4.2,0) {$1$};
    \node at (-.2,4) {$1$};
         \node[fill,circle,  minimum size=3pt, inner sep=0pt, sexyRed ]
             (X14) at (1.4, 2.4){};
         \node[fill,circle,  minimum size=3pt, inner sep=0pt, sexyRed ]
             (X14) at (2.4, 1.4){};

    \path[draw, thick] (5,0) -- (9,0) -- (9,4) -- (5,4) --cycle; 
    \node at (4.2,0) {$0$};
    \node at (9.2,0) {$1$};
    \node at (9.2,4) {$1$};
    \fill[pattern=north west lines, pattern color=lightBlue] (7,2)  rectangle (9,4);
    \node at (8,3) {$A$};
\end{tikzpicture}
\end{center}
\end{figure}

        
\end{frame}
\begin{frame}[t]
    \frametitle{Axiomes de probabilités}
           \begin{block}{Axiomes}
               
               \begin{enumerate}
                   \small
                   \item Positivité:
                       \begin{equation}
                          \alert{P(A) \geq 0} 
                       \end{equation}
                    \item Normalité:

                       \begin{equation}
                          \alert{P(\Omega) = 1} 
                       \end{equation}

                   \item Additivité

                       \begin{equation}
                           \alert{
                           \text{si } A \cap B = \emptyset \text{, alors }
                           P(A\cup B) = P(A) + P(B)}
                       \end{equation}
               \end{enumerate}
           \end{block}
           
\end{frame}
% }}} Axiom de probabilite %
\begin{frame}[t]
    \frametitle{Résultats directes}
    \begin{columns}
        \small
        \begin{column}{0.4\textwidth}
            \alert{\textbf{Axiomes}}
            \begin{itemize}
                \item $P(A) \geq 0$
                    \vspace*{1cm}
                \item $P(\Omega)  = 1 $
                    \vspace*{1cm}

                \item $P(A\cup B) = P(A) + P(B)  $
            \end{itemize}

        \end{column}
        \begin{column}{0.6\textwidth}
            \small
            \alert{\textbf{Conséquences}}

            \begin{itemize}
                \item $P(A) \leq 1$
                    \vspace*{1cm}
                \item $P(\emptyset) = 0$

                    \vspace*{1cm}

                \item $P(A\cup B \cup C) = P(A) + P(B)  + P(C) $
            \end{itemize}
        \end{column}
    \end{columns}

    \vspace*{1cm}
        Plus Généralement, pour $k$ évènements disjoints:
        \begin{equation}
            P(\bigcup A_k)  = \sum_k P(A_k)
        \end{equation}
    
\end{frame}

% Autres consequences {{{ %
\begin{frame}[t]
    \frametitle{Autres conséquences}
    
    \begin{itemize}
        \item si $ A \subset B$, alors $ P(A) \leq P(B)$
        \vspace*{2cm}
    \item $P(A\cup B) = P(A) + P(B) - P(A\cap B)$
        \vspace*{2cm}
    \item $P(A \cup B) = P(A) + P(B)$
    \end{itemize}
\end{frame}
% }}}  %
% Plus de conséquences {{{ %
\begin{frame}[t]
    \frametitle{Conséquences}
    \begin{itemize}

        \item $P(A\cup B \cup C) = P(A) + P(A^c \cap B) + P(A^c\cap B^c \cap C)$
    \end{itemize}
\end{frame}

% }}}  Plus de conséquences %

% Examples {{{ %
\begin{frame}[t]
    \frametitle{Exemples}
    \begin{columns}
        \begin{column}{0.5\textwidth}
            \small Lance d'un de a quatre faces.

            \begin{figure}[htpb]
            \begin{center}
            \begin{tikzpicture}[scale=1, transform shape]
                
                \draw[thick] (0,0) grid (4,4);
                \foreach \x/\y in {0.5/1, 1.5/2, 2.5/3, 3.5/4}
                {
                    \node at (\x,-.2) {\y};
                    \node at (-.2,\x) {\y};
                }
                \only<2->
                {
                    \fill[pattern = crosshatch, pattern color=sexyRed] (0,0)
                        rectangle (1,4);
                }

                \only<4->
                {
                    \fill[pattern = crosshatch, pattern color=lightBlue] (3,3)
                        rectangle (4,4);

                }

                \only<6->
                {
                    \fill[pattern = dots, pattern color=green] (1,1) rectangle (2,4);

                    \fill[pattern = dots, pattern color=green] (1,1)
                        rectangle (4,2);

                }
            \end{tikzpicture}
            \end{center}
            \end{figure}
        \end{column}
        \begin{column}{0.5\textwidth}
           \begin{itemize}
               \item<1-> $P(X = 1) = $
               \item<3-> Soit $Z = \min(x,y)$
                \item<3-> $P(Z = 4) = $ 
                \item<5-> $P(Z = 2) = $
                
           \end{itemize} 
            
        \end{column}
    \end{columns}
\end{frame}
% }}} Examples %
% ----- Discrete uniforme {{{ %
\subsection{Exemple discret uniforme}
\begin{frame}[t]
    \frametitle{Exemple discret uniforme}

    \begin{columns}
        \begin{column}{0.5\textwidth}
            
            \begin{itemize}
                \scriptsize
                \item On assume que \alert{$\Omega$} consiste de $\mathbf{n}$.
                \item Tous les éléments ont  la \textbf{même probabilité}.
                \item L'évènement \alert{$A$} contient $\mathbf{k}$ éléments.
                    \pause
                    \begin{equation}
                        P(A) = \dfrac{k}{n}
                    \end{equation}

            \end{itemize}
        \end{column}
        \begin{column}{0.5\textwidth}
            \begin{figure}[htpb]
            \begin{center}
            \begin{tikzpicture}[scale=1, transform shape]
                \draw[fill=sexyRed, thick,opacity=0.6] (0,0) rectangle (4,3);
                \foreach \x/\y in {.5/.7, .3/2.3, 1/2.3, 1.3/2, 2/.9, 2.1/2.5,
                2.5/1.5, 2.6/.8, 3.1/2.7, 2.8/.8  }
                {

    \node[fill,circle,  minimum size=3pt, inner sep=0pt] at (\x, \y){};
                }

                \node [fill=lightBlue,opacity=0.7,ellipse,draw=black, thick,
                    minimum width=2cm, minimum height=1cm] at (2.5,.7){};

                \node (T) at (2, -1.5) {\small prob = $\frac{1}{n}$};
                \coordinate (A)  (0.3,2.7);
                \path[draw,->,>=stealth](2,-1.2)--(.5,0.7);
                \path[draw,->,>=stealth](2,-1.2)--(2.6,0.8);
                \path[draw,->,>=stealth](2,-1.2)--(2.8,0.8);
            \end{tikzpicture}

            \end{center}
            \end{figure}
            \end{column}
    \end{columns}
    
\end{frame}
% }}} ----- %

% Conti9nue {{{ %
\begin{frame}[t]
    \frametitle{Exemple continu}
    \begin{columns}
        \begin{column}{0.5\textwidth}

    \begin{figure}[htpb]
    \begin{center}
    \begin{tikzpicture}[scale=1, transform shape]
        
        \path[draw,thick,-stealth] (0,0)--(5,0)node[pos=1,label=below:$x$]{};
        \path[draw,thick,-stealth] (0,0)--(0,5)node[pos=1,label=below:$y$]{};
        \node (A) at (4,-0.2) {$1$};
        \node (B)at (-0.2,4) {$1$};
        \path[draw, thick] (A) -- (4,4) -- (B);
        
     \path[draw, pattern = grid, pattern color=sexyRed] (0,0) --(2.5,0) -- (0,2.5) -- cycle;
        
    \end{tikzpicture}
    \end{center}
    \end{figure}
        \end{column}
        \begin{column}{0.5\textwidth}
            \begin{itemize}
                \small
                \item \alert{ Simple exemple uniforme: } probabilité = surface.
                \item \begin{equation*}
                    P\left( \left\{(x,y)\;|\; x + y \leq \frac{1}{2}\right\}\right)
                \end{equation*}
            \item
                \begin{equation*}
                   P\left(\left\{(0.5,0.3)\right\}\right) = 
                \end{equation*}
            \end{itemize}
            
        \end{column}
    \end{columns}
    
\end{frame}
% }}} Conti9nue %

% Etape de calcul de probabilité {{{ %
\begin{frame}[t]
    \frametitle{Étapes de calcul de probabilité}
\begin{enumerate}
    \small
    \item Spécifier l'espace des \textbf{États}\\[.5cm]
    \item Spécifier la \textbf{loi de probabilité}!!\\[.5cm]
    \item Identifier l'\textbf{évènement}.\\[1cm]
    \item Calculer...

\end{enumerate}
    
\end{frame}
% }}}  Etape de calcul de probabilité %

% Exemple Discret Infini {{{ %
\begin{frame}[t]
    \frametitle{Exemple discret et Infini}
    \begin{columns}
        \begin{column}{0.6\textwidth}
            \begin{itemize}
                \small
                \item Espace d'états $\{1,2, \ldots\}$
                    \pause
                \item la loi de probabilité est donnée par:
                    \begin{equation*}
                       P(n) = \dfrac{1}{2^n} 
                    \end{equation*}
                    \pause
                    \begin{itemize}
                        \item S'agit il d'une loi de probabilité?
                            \vspace*{2cm}
                        \item Calculer $P( \text{ résultat pair}) = 

                    \end{itemize}
            \end{itemize}
            
        \end{column}
        \begin{column}{0.4\textwidth}
            \begin{figure}[htpb]
            \begin{center}
            \begin{tikzpicture}[ xscale=0.7, transform shape]
                \draw[thick,->,>=stealth] (0,0) -- (6,0);
                \draw[thick,->,>=stealth] (0,0) -- (0,2);
                \node at (-.2,2) {$p$};
                \foreach \x in {1, 2, 3, 4, 5}
                {

                 \node[fill,circle,  minimum size=3pt, inner sep=0pt]
                   at (\x, 0){};

                 \node[draw=black, fill=blue!30,circle,  minimum size=3pt, inner sep=0pt]
                   at (\x, 1./\x){};
                 \path[dotted,draw] (\x, 0) -- (\x, 1./\x);

                 \node at (\x, -.2) {\scriptsize \x};
                 
                }

                
            \end{tikzpicture}
            \end{center}
            \end{figure}
            
        \end{column}
    \end{columns}
    
\end{frame}
% }}} Exemple Discret Infini %


% Somme infinie {{{ %
\begin{frame}[t]
    \frametitle{Somme Infinie}

    \begin{itemize}
        \small
        \item Resultat plus fort que la somme:
            \begin{block}{Theroeme}
               Si $A_1$, $A_2$, $A_3$, $\ldots$ est une \alert{suite}
               d'évènements  \alert{disjoints}, alors:

               \begin{equation}
                   P(A_1\cup P_2 \cup A_3\cup \ldots) = P(A_1) + P(A_2) + P(A_3)
                   + \ldots.
               \end{equation}
            \end{block}
        \item Faites attention que la somme doit etre \textbf{dénombrable}. \end{itemize}
    
\end{frame}
% }}} Somme infinie %
\end{document}
