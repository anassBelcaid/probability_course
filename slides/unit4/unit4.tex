%{{{
\documentclass{beamer}
\usetheme{ensam}
\usepackage{pgfplots}
\usepackage{subcaption}
\usepackage{forest}
\usepackage{acronym}
\usepackage{tikz}
\usetikzlibrary{calc}
\usepackage{amsmath}
\usepackage {algorithmic}
\usepackage{algorithm}
\usepackage{eqparbox}
\usepackage[font=scriptsize]{caption}
\usetikzlibrary{bayesnet,positioning,calc}
\tikzstyle{obs} = [latent,fill=lightBlue]
\tikzstyle{default}=[draw=sexyRed,thick,rounded corners,text width=0.5in,font=\scriptsize,align=center]
\usepgfplotslibrary{colorbrewer}
\definecolor{ForestGreen}{RGB}{34,139,34}
\newcommand{\comment}[1]{\textcolor{ForestGreen}{#1}}
%algorithmic comment
\renewcommand\algorithmiccomment[1]{%
  \hfill\comment{\#\scriptsize\eqparbox{COMMENT}{#1}}%
}
\renewcommand{\algorithmicrequire}{\textbf{Input:}}
\renewcommand{\algorithmicensure}{\textbf{Output:}}
\title{Problèmes résolus en conditionnement}
\author{\underline{A.Belcaid}}
\institute{\small ENSA-Safi} 

%tikz bayesian theme
\usetikzlibrary{bayesnet,positioning,calc}
\tikzstyle{obs} = [latent,fill=lightBlue]
\tikzstyle{default}=[draw=sexyRed,thick,rounded corners,text width=0.5in,font=\scriptsize,align=center]
\DeclareMathOperator{\argmin}{argmin}

\pgfplotsset{every tick label/.append style={font=\tiny}}


%}}}

\begin{document}
\maketitle

\begin{frame}
\tableofcontents
\end{frame}

\begin{frame}[t]
  \frametitle{Trounoi d'echecs}
  \begin{block}{Exercice}
    \small
    Un tournoi d'échecs est organisé selon les règles suivantes:
    \begin{enumerate}
      \item \textbf{Bo} et \textbf{Ci},les deux challengeurs de l'année dernière,
      jeuent a un match de \textbf{deux parties}.
    \item Si l'un deux gagnent les deux parties, il joue un autre match de
      \textbf{deux parties} contre
      \textbf{AI} le teneur du titre.
    \item Dans le match contre \textbf{AI}, il retient son titre en cas de gain
      ou d'égalité.
    \item La challengeur retient le titre s'il bat \textbf{AI} dans les deux
      parties.
    \end{enumerate}
  \end{block}

    On as les donnes suivantes:
    \begin{itemize}
      \item Bo peut vaincre  CI avec une urobiline $\mathbf{0.6}$.
      \item Ai peut vaincre Bo avec une probabilité $\mathbf{0.5}$.
      \item Ai peut vaincre Ci avec un probabilité $\mathbf{0.7}$.
    \end{itemize}
\end{frame}

\begin{frame}[t]
  \frametitle{Question}
  \begin{block}{Questions}
\small
 \begin{enumerate}
   \item Déterminer les probabilités que:
     \begin{itemize}
       \item Le match sera décide dans le deuxième rond.
       \item  Bo va gagner le premier rond.
       \item Ai retient son titre.
     \end{itemize}
    \item Maintenant on sait qu'a atteint le deuxième rond.
      \begin{itemize}
        \item Calculer la probabilité que Bo qui va jouer contre Ai.
        \item Ai retient son titre.
      \end{itemize}
    \item Sachant maintenant qu'on as atteint le deuxième rond, et qu'il est
      decide dans un seul match.
      \begin{itemize}
        \item Quelle est la probabilité que c'est Bo qui a gagne.
      \end{itemize}
 \end{enumerate}   
  \end{block}
\end{frame}

  \begin{frame}[t]
    \frametitle{Problème Monty Hall }
   \begin{block}{Problème Monty Hall}
    \scriptsize
    Le problème classique \alert{\textbf{Monty Hall}}  est un problème classique
    trouvant ces racines d'un \textbf{jeu televise Americain}. Dans ce jeu, on
    vous présente \textbf{trois} portes. On vous promet que l'une de ces portes
    contient une \alert{recompense}.\\[4pt]
    \begin{enumerate}
      \scriptsize
      \item Vous pointez le doigt sur une première porte.\\[4pt]
      \item Une fois choisi, l'animateur ouvre l'\textbf{une des deux
        portes}\\[4pt]
      \item On vous demande alors, si vous voulez préservez votre choix, ou le
        \textbf{changer}.
        restantes.
    \end{enumerate}
   \end{block} 
    On considère alors les deux stratégies:

    \begin{enumerate}
      \scriptsize
      \item Toujours garder le premier choix. ( i.e ne pas changer)
      \item Changer a l'autre porte fermée.
    \end{enumerate}
    \centering
    \alert{\textbf{Quelle est la meilleure stratégie?}} 
  \end{frame}

  \begin{frame}[t]
    \frametitle{Marche aléatoire}
    
    \begin{block}{Marche aléatoire}
      \scriptsize
      On imagine une personne \textbf{ivre} qui se déplace tout au lent d'une
      \alert{\textbf{ligne droite}}. Il arrive a maintenir sa balance, cependant
      il contrôle par sa direction (i.e Il peut avancer comme il peut
      reculer).\\[4pt]

      \begin{itemize}
        \scriptsize
        \item On suppose qu'il peut avancer avec un probabilité $\mathbf{p}$.
        \item Ainsi il peut reculer par une probabilité $\mathbf{1-p}$.
      \end{itemize}
    \end{block}

    \vspace*{.5cm}
    \begin{enumerate}
      \scriptsize
      \item Calculer la probabilité qu'après  \alert{\textbf{deux pas}}, cet
        ivre sera a sa \textbf{position initiale}.\\[4pt]
      \item Quelle est la probabilité qu'après \alert{\textbf{trois pas}}, il
        sera un \textbf{pas devant}?\\[4pt]
      \item Maintenant on suppose qu'après \textbf{trois pas}, il as
        effectivement terminé devant.
        \begin{itemize}
          \scriptsize
          \item Quelle est la probabilité que le premier pas est
            \textbf{devant}?
        \end{itemize}

    \end{enumerate}
  \end{frame}
\end{document}
